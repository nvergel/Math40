\documentclass[11pt,letterpaper,boxed]{hmcpset}

% include this package to customize page layout of document (e.g. to adjust margins)
\usepackage[margin=0.9in]{geometry}
\usepackage{ulem}

% include this to import graphics files with command \includegraphics
\usepackage{graphicx, url}

\usepackage{mathtools}
\usepackage{esvect}

% set numbering style for enumerated lists to be of form (a), (b), (c), etc.
\renewcommand{\labelenumi}{{\bf (\alph{enumi})}}

\newcommand{\ZZ}{\mathbb{Z}}
%\newcomand{\iff}{\Longleftrightarrow}
\newcommand{\ord}{\mbox{ord}}
\renewcommand{\mod}[3]{#1 \equiv #2 \ (\mbox{mod } #3)}
\addtolength{\fboxsep}{3px}
\renewcommand{\gcd}{\mbox{gcd}}
\newcommand{\ul}[1]{\underline{#1}}


% info for header block in upper right hand corner
\class{Math 40, Section: $\ \ \ $  } 		%% replace spaces $\ \ \ $ with section #
\assignment{Homework \#1}

\begin{document}

%-----------------------------------------------------------------------------------------------------------------------%

\begin{problem}[\#2] 
Draw the vectors in Exercise 1 with their tails at the point (2, -3).
\end{problem}

%\begin{solution}

%\end{solution}

\pagebreak

%-----------------------------------------------------------------------------------------------------------------------%

\begin{problem}[\#5] 
For each of the following pairs of points, draw the vector $\vv{AB}$. Then compute and redraw $\vv{AB}$ as a vector in standard position.
\begin{enumerate}
\item $A=(1,-1), B=(4,2)$
\item $A=(0, -2), B=(2,-1)$
\item $A=(2,\frac32), B=(\frac12,3)$
\item $A=(\frac13,\frac13), B=(\frac16,\frac12)$
\end{enumerate}
\end{problem}

%\begin{solution}

%\end{solution}

\pagebreak

%-----------------------------------------------------------------------------------------------------------------------%

\begin{problem}[\#8] 
Refer to the vectors in Exercise 1. Compute the indicated vectors and also show how the results can be obtained geometrically.\\
\[
\textbf{b}-\textbf{c}
\]
\end{problem}

%\begin{solution}

%\end{solution}

\pagebreak

%-----------------------------------------------------------------------------------------------------------------------%

\begin{problem}[\#12] 
Refer to the vectors in Exercise 3. Compute the indicated vectors.\\
\[
3\textbf{b}-2\textbf{c}+\textbf{d}
\]
\end{problem}

%\begin{solution}

%\end{solution}

\pagebreak

%-----------------------------------------------------------------------------------------------------------------------%

\begin{problem}[\#14] 
In Figure 1.24, A, B, C, D, E, and F are the vertices of a regular hexagon centered at the origin.\\
Express each of the following vectors in terms of\\
$\textbf{a}=\vv{OA}$ and $\textbf{b}=\vv{OB}$:
\begin{enumerate}
\item $\vv{AB}$
\item $\vv{BC}$
\item $\vv{AD}$
\item $\vv{CF}$
\item $\vv{AF}$
\item $\vv{BC}+\vv{DE}+\vv{FA}$
\end{enumerate}
\end{problem}

%\begin{solution}

%\end{solution}

\pagebreak

%-----------------------------------------------------------------------------------------------------------------------%

\begin{problem}[\#16] 
Simplify the given vector expression.\\
Indicate which properties in Theorem 1.1 you use.\\
\[
-3(\textbf{a}-\textbf{c})+2(\textbf{a}+2\textbf{b})+3(\textbf{c}-\textbf{b})
\]
\end{problem}

%\begin{solution}

%\end{solution}

\pagebreak

%-----------------------------------------------------------------------------------------------------------------------%

\begin{problem}[\#17] 
Solve for the vector \textbf{x} in terms of the vectors \textbf{a} and \textbf{b}.\\
\[
\textbf{x} - \textbf{a} = 2(\textbf{x} - 2\textbf{a})
\]
\end{problem}

%\begin{solution}

%\end{solution}

\pagebreak

%-----------------------------------------------------------------------------------------------------------------------%

\begin{problem}[\#22] 
Draw the standard coordinate axes on the same diagram as the axes relative to \textbf{u} and \textbf{v}. Use these to find \textbf{w} as a linear combination of \textbf{u} and \textbf{v}.\\
$\textbf{u} = \begin{bmatrix} -2 \\ 3 \end{bmatrix},
\textbf{v} = \begin{bmatrix} 2 \\ 1 \end{bmatrix},
\textbf{w} = \begin{bmatrix} 2 \\ 9 \end{bmatrix}$
 
\end{problem}

%\begin{solution}

%\end{solution}

\pagebreak

%-----------------------------------------------------------------------------------------------------------------------%

\end{document}